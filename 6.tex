GMF是一个复杂的试验状态,对于圆管试件,温度梯度导致试件径向与轴向产生额外应力,试件内表面的温度低,表现为多轴拉伸载荷,与之相反,试件外表面的温度高,表现为多轴压缩载荷。
同时由于温度梯度的存在,导致材料的力学性能在径向方向有所区别,这会导致材料在试验过程中的塑性应变累积有所不同,我们无法通过试验观测,只能通过合适的本构模型进行计算。
通过对疲劳断口的观察,我们发现裂纹的萌生都是由外表面开始的,
如图\ref{Fig:plot_exp_fatigue_life}所示,我们发现对于无涂层的试件,TGMF的寿命要明显低于TMF的寿命,更大大低于恒温疲劳的寿命。

(1)W-B模型


(2)F-S模型
The following models have been selected to the analysis:
 non-energetic stress-strain Fatemi-Socie model, based on the critical plane approach [22,23]

与应力准则相对应,早期的研究者Yokobori等将静强度理论应用到多轴应变准则中来。
采用最大法向应变、最大剪应变、Von Mises等效应变等代替Manson-Coffin公式的应变,与单轴低周疲劳估算相似,进行多轴疲劳寿命预算。
由于该方法没有考虑不同应力路径对材料响应和疲劳寿命的影响,因而不能用于预测多轴非比例加载下的疲劳寿命。

Brown和Miller[31]根据疲劳裂纹萌生和扩展的物理解释提出了一种多轴疲劳理论,与Findley[17-20]提出的应力准则相类似,Brown和Miller认为最大剪平面上的剪应变和法向应变这两个参数都应该考虑。他们提出裂纹第一阶段沿最大剪切面生成,第二阶段沿垂直于最大拉应变方向扩展。
Brown和Miller准则由一系列由最大剪应变${\gamma _{\max }}$和最大剪应变平面上的法向应变${\varepsilon _n}$为坐标所组成的$\Gamma$平面上的等寿命曲线组成,对于给定寿命有
\[{\gamma _{\max }} = f\left( {{\varepsilon _n}} \right)\]

上式对于A、B两类裂纹有着不同的表达式。Brown和Miller将裂纹分为两种情况。在复合拉伸和扭转中,主应变${\varepsilon _1}$和${\varepsilon _3}$平行于表面,裂纹沿着表面扩展称为A类裂纹;对于正的双向拉伸,应变${\varepsilon _3}$垂直于自由表面,裂纹在自由表面上萌生进而沿纵深方向扩展称为B类裂纹。
Kandil,Brown和Miller[32]给出A类裂纹表达式的简化形式:
\[\frac{{\Delta {\gamma _{\max }}}}{2} + k\Delta {\varepsilon _n} = C\]
其中,$\Delta {\varepsilon _n}$是临界面上的法向应变变幅,$k$和$C$是材料常数。这一准则被广泛讨论及应用[33-36]。

Wang和Brown[37]用相邻两个最大剪应变折返点之间的法向应变变程${\varepsilon _n^*}$来代替法向应变变程$\Delta {\varepsilon _n}$,给出疲劳破坏模型
\[\frac{{\Delta {\gamma _{\max }}}}{2} + k\varepsilon _n^* = C\]
其中$k$和$C$是材料常数。

Fatemi和Socie[38]研究后发现,由于Kandil、Brown和Miller破坏模型的参数都是应变,没有考虑非比例加载下由于主轴旋转所产生的附加强化效
应,所以预测的多轴非比例加载下的疲劳寿命偏于危险,建议以最大剪应变平面上的最大法向应力${\sigma _{n,\max }}$
代替法向应变变程$\Delta {\varepsilon _n}$作为参数,提出准则如下:
\[\frac{{\Delta {\gamma _{\max }}}}{2}\left( {1 + k\frac{{{\sigma _{n,\max }}}}{{{\sigma _y}}}} \right) = C\]

\subsection{Brown-Miller Model}

1973年Brown和Miller\cite{Brown1973}根据疲劳裂纹扩展的物理解释提出了一种多轴疲劳模型,以最大剪切应变$\Delta\gamma_{max}$和最大剪切应变平面上的法向应变$\varepsilon_{n,max}$作为疲劳损伤参量。
1977年Kanazawa等\cite{KANAZAWA1979}对一批不锈钢进行了不同幅值比值和相位角的双轴拉扭试验,对最大剪应变平面上的正应变幅进行了观察,认为其面上正应变和剪应变之间的相位差对疲劳寿命没影响,故采用正应变幅和剪应变幅作为寿命拟合参数。
1993年Wang C. H.、Brown M. W.提出了一种考虑最大剪应变幅和最大剪应变幅平面上正应变程的多轴疲劳寿命模型,仅一个试验参数,可由单轴拉伸试验和扭转试验来确定[155]。
其考虑正应变变程的寿命计算公式(简称WB模型)为:
\[\frac{{\Delta {\gamma _{{\rm{max}}}}}}{2} + S{\varepsilon _{{\rm{n,max}}}} = \left[ {1 + {\nu _{\rm{e}}} + \left( {1 + {\nu _{\rm{e}}}} \right)S} \right]\frac{{{{\sigma '}_f}}}{E}{\left( {2{N_f}} \right)^b} + \left[ {1 + {\nu _{\rm{p}}} + \left( {1 + {\nu _{\rm{p}}}} \right)S} \right]{\varepsilon '_f}{\left( {2{N_f}} \right)^c}\]
其中,$\Delta\gamma_{max}$为最大剪应变平面上的剪应变幅,$\epsilon_{n,max}$为垂直最大剪应变平面上的最大正应变程,$S$为材料常数,$\nu_e$为弹性泊松比,$\nu_p$为塑性泊松比。
一般地,参数$S$不是常数,它会随疲劳寿命不同而改变。