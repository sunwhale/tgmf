\documentclass{article}

\usepackage{ctex}
\newcommand{\marked}[1]{\textcolor{red}{#1}}

\begin{document}
\title{镍基高温合金热梯度机械疲劳条件下的寿命预测}

\author{袁荒,孙经雨}
\date{\today}
\maketitle

\section*{摘要}

\marked{Background/Introduction:}

对于燃气轮机,内部冷却是降低零件温度,保证零件正常运转的重要手段。有研究表明,温度梯度对零件的应力分布和疲劳寿命有着很重要的影响。

\marked{Subject of the paper:}

热梯度机械载荷作用下材料的疲劳寿命预测对于燃气轮机热端部件的设计有着重要的意义。

\marked{Techniques and methods used:}

本文针对镍基高温合金Inconel 718,设计和开展中心冷却的镍基高温合金热梯度机械疲劳(TGMF)试验,获取材料在热梯度机械疲劳载荷作用下的应力应变曲线与疲劳寿命。

\marked{Main Results:}

温度梯度会导致应力场的变化,通过有限元计算,确定试件在每个循环内的温度场分布,同时基于循环塑性本构方程计算当前温度场下的应力应变曲线。
试验表明与无温度梯度的热机械疲劳(TMF)寿命相比,TGMF的寿命要明显降低。
考虑温度与机械载荷相位对TGMF寿命的影响,在低周疲劳的情况下,TGMF同相位(IP)的寿命要明显低于反相位(OP)的寿命。
同时研究涂覆热障涂层(TBC)对镍基高温合金的热梯度机械疲劳性能的影响,结果表明,涂覆有热障涂层的试样寿命高于无涂层的试样寿命,但仍低于TMF试样寿命。

\marked{Conclusions:}

选取多种多轴疲劳模型对试验数据进行预测,误差较大,本文引入温度梯度的修正项,建立适用于热梯度机械疲劳的寿命预测模型。


\end{document} 