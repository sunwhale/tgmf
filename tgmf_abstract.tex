\documentclass{article}

\usepackage{ctex}


\begin{document}
\title{镍基高温合金热梯度机械疲劳条件下的寿命预测}

\author{袁荒,孙经雨}
\date{\today}
\maketitle

\section*{摘要}
对于燃气轮机,内部冷却是降低零件温度,保证零件正常运转的重要手段。
设计和开展中心冷却的镍基高温合金热梯度机械疲劳(TGMF)试验。
与无温度梯度的热机械疲劳(TMF)寿命相比,TGMF的寿命要明显降低。
考虑温度与机械载荷相位对TGMF寿命的影响,在低周疲劳的情况下,TGMF同相位(IP)的寿命要明显低于反相位(OP)的寿命。
同时研究涂覆热障涂层(TBC)对镍基高温合金的热梯度机械疲劳性能的影响,结果表明,涂覆有热障涂层的试样寿命高于无涂层的试样寿命,但仍低于TMF试样寿命。
温度梯度会导致应力场的变化,通过有限元计算,确定试件在每个循环内的温度场分布,同时基于循环塑性本构方程计算当前温度场下的应力应变曲线。
选取多种多轴疲劳模型对试验数据进行预测,误差较大,引入温度梯度的修正项,建立热梯度机械疲劳寿命预测模型。
\end{document} 